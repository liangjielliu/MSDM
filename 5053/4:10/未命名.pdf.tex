% Options for packages loaded elsewhere
\PassOptionsToPackage{unicode}{hyperref}
\PassOptionsToPackage{hyphens}{url}
%
\documentclass[
]{article}
\usepackage{amsmath,amssymb}
\usepackage{iftex}
\ifPDFTeX
  \usepackage[T1]{fontenc}
  \usepackage[utf8]{inputenc}
  \usepackage{textcomp} % provide euro and other symbols
\else % if luatex or xetex
  \usepackage{unicode-math} % this also loads fontspec
  \defaultfontfeatures{Scale=MatchLowercase}
  \defaultfontfeatures[\rmfamily]{Ligatures=TeX,Scale=1}
\fi
\usepackage{lmodern}
\ifPDFTeX\else
  % xetex/luatex font selection
\fi
% Use upquote if available, for straight quotes in verbatim environments
\IfFileExists{upquote.sty}{\usepackage{upquote}}{}
\IfFileExists{microtype.sty}{% use microtype if available
  \usepackage[]{microtype}
  \UseMicrotypeSet[protrusion]{basicmath} % disable protrusion for tt fonts
}{}
\makeatletter
\@ifundefined{KOMAClassName}{% if non-KOMA class
  \IfFileExists{parskip.sty}{%
    \usepackage{parskip}
  }{% else
    \setlength{\parindent}{0pt}
    \setlength{\parskip}{6pt plus 2pt minus 1pt}}
}{% if KOMA class
  \KOMAoptions{parskip=half}}
\makeatother
\usepackage{xcolor}
\usepackage[margin=1in]{geometry}
\usepackage{color}
\usepackage{fancyvrb}
\newcommand{\VerbBar}{|}
\newcommand{\VERB}{\Verb[commandchars=\\\{\}]}
\DefineVerbatimEnvironment{Highlighting}{Verbatim}{commandchars=\\\{\}}
% Add ',fontsize=\small' for more characters per line
\usepackage{framed}
\definecolor{shadecolor}{RGB}{248,248,248}
\newenvironment{Shaded}{\begin{snugshade}}{\end{snugshade}}
\newcommand{\AlertTok}[1]{\textcolor[rgb]{0.94,0.16,0.16}{#1}}
\newcommand{\AnnotationTok}[1]{\textcolor[rgb]{0.56,0.35,0.01}{\textbf{\textit{#1}}}}
\newcommand{\AttributeTok}[1]{\textcolor[rgb]{0.13,0.29,0.53}{#1}}
\newcommand{\BaseNTok}[1]{\textcolor[rgb]{0.00,0.00,0.81}{#1}}
\newcommand{\BuiltInTok}[1]{#1}
\newcommand{\CharTok}[1]{\textcolor[rgb]{0.31,0.60,0.02}{#1}}
\newcommand{\CommentTok}[1]{\textcolor[rgb]{0.56,0.35,0.01}{\textit{#1}}}
\newcommand{\CommentVarTok}[1]{\textcolor[rgb]{0.56,0.35,0.01}{\textbf{\textit{#1}}}}
\newcommand{\ConstantTok}[1]{\textcolor[rgb]{0.56,0.35,0.01}{#1}}
\newcommand{\ControlFlowTok}[1]{\textcolor[rgb]{0.13,0.29,0.53}{\textbf{#1}}}
\newcommand{\DataTypeTok}[1]{\textcolor[rgb]{0.13,0.29,0.53}{#1}}
\newcommand{\DecValTok}[1]{\textcolor[rgb]{0.00,0.00,0.81}{#1}}
\newcommand{\DocumentationTok}[1]{\textcolor[rgb]{0.56,0.35,0.01}{\textbf{\textit{#1}}}}
\newcommand{\ErrorTok}[1]{\textcolor[rgb]{0.64,0.00,0.00}{\textbf{#1}}}
\newcommand{\ExtensionTok}[1]{#1}
\newcommand{\FloatTok}[1]{\textcolor[rgb]{0.00,0.00,0.81}{#1}}
\newcommand{\FunctionTok}[1]{\textcolor[rgb]{0.13,0.29,0.53}{\textbf{#1}}}
\newcommand{\ImportTok}[1]{#1}
\newcommand{\InformationTok}[1]{\textcolor[rgb]{0.56,0.35,0.01}{\textbf{\textit{#1}}}}
\newcommand{\KeywordTok}[1]{\textcolor[rgb]{0.13,0.29,0.53}{\textbf{#1}}}
\newcommand{\NormalTok}[1]{#1}
\newcommand{\OperatorTok}[1]{\textcolor[rgb]{0.81,0.36,0.00}{\textbf{#1}}}
\newcommand{\OtherTok}[1]{\textcolor[rgb]{0.56,0.35,0.01}{#1}}
\newcommand{\PreprocessorTok}[1]{\textcolor[rgb]{0.56,0.35,0.01}{\textit{#1}}}
\newcommand{\RegionMarkerTok}[1]{#1}
\newcommand{\SpecialCharTok}[1]{\textcolor[rgb]{0.81,0.36,0.00}{\textbf{#1}}}
\newcommand{\SpecialStringTok}[1]{\textcolor[rgb]{0.31,0.60,0.02}{#1}}
\newcommand{\StringTok}[1]{\textcolor[rgb]{0.31,0.60,0.02}{#1}}
\newcommand{\VariableTok}[1]{\textcolor[rgb]{0.00,0.00,0.00}{#1}}
\newcommand{\VerbatimStringTok}[1]{\textcolor[rgb]{0.31,0.60,0.02}{#1}}
\newcommand{\WarningTok}[1]{\textcolor[rgb]{0.56,0.35,0.01}{\textbf{\textit{#1}}}}
\usepackage{graphicx}
\makeatletter
\newsavebox\pandoc@box
\newcommand*\pandocbounded[1]{% scales image to fit in text height/width
  \sbox\pandoc@box{#1}%
  \Gscale@div\@tempa{\textheight}{\dimexpr\ht\pandoc@box+\dp\pandoc@box\relax}%
  \Gscale@div\@tempb{\linewidth}{\wd\pandoc@box}%
  \ifdim\@tempb\p@<\@tempa\p@\let\@tempa\@tempb\fi% select the smaller of both
  \ifdim\@tempa\p@<\p@\scalebox{\@tempa}{\usebox\pandoc@box}%
  \else\usebox{\pandoc@box}%
  \fi%
}
% Set default figure placement to htbp
\def\fps@figure{htbp}
\makeatother
\setlength{\emergencystretch}{3em} % prevent overfull lines
\providecommand{\tightlist}{%
  \setlength{\itemsep}{0pt}\setlength{\parskip}{0pt}}
\setcounter{secnumdepth}{-\maxdimen} % remove section numbering
\usepackage{bookmark}
\IfFileExists{xurl.sty}{\usepackage{xurl}}{} % add URL line breaks if available
\urlstyle{same}
\hypersetup{
  pdftitle={Starbucks Log Return Analysis},
  hidelinks,
  pdfcreator={LaTeX via pandoc}}

\title{Starbucks Log Return Analysis}
\author{}
\date{\vspace{-2.5em}}

\begin{document}
\maketitle

\subsection{📁
数据读取与预处理}\label{ux6570ux636eux8bfbux53d6ux4e0eux9884ux5904ux7406}

我们分析 Starbucks 股票与 S\&P500 指数的对数收益率数据。文件为
\texttt{d-sbuxsp0106.txt},包含日期、SBUX 简单收益率、SP500
简单收益率三列。

\begin{Shaded}
\begin{Highlighting}[]
\CommentTok{\# 读取数据}
\NormalTok{df }\OtherTok{\textless{}{-}} \FunctionTok{read.table}\NormalTok{(}\StringTok{"d{-}sbuxsp0106.txt"}\NormalTok{, }\AttributeTok{header =} \ConstantTok{FALSE}\NormalTok{)}
\FunctionTok{colnames}\NormalTok{(df) }\OtherTok{\textless{}{-}} \FunctionTok{c}\NormalTok{(}\StringTok{"Date"}\NormalTok{, }\StringTok{"SBUX\_simple"}\NormalTok{, }\StringTok{"SP500\_simple"}\NormalTok{)}

\CommentTok{\# 转换日期格式并按时间排序}
\NormalTok{df}\SpecialCharTok{$}\NormalTok{Date }\OtherTok{\textless{}{-}} \FunctionTok{as.Date}\NormalTok{(}\FunctionTok{as.character}\NormalTok{(df}\SpecialCharTok{$}\NormalTok{Date), }\AttributeTok{format =} \StringTok{"\%Y\%m\%d"}\NormalTok{)}
\NormalTok{df }\OtherTok{\textless{}{-}}\NormalTok{ df[}\FunctionTok{order}\NormalTok{(df}\SpecialCharTok{$}\NormalTok{Date), ]}
\end{Highlighting}
\end{Shaded}

\subsection{📈 计算对数收益率(Log
Return)}\label{ux8ba1ux7b97ux5bf9ux6570ux6536ux76caux7387log-return}

\begin{Shaded}
\begin{Highlighting}[]
\NormalTok{df}\SpecialCharTok{$}\NormalTok{SBUX\_log }\OtherTok{\textless{}{-}} \DecValTok{100} \SpecialCharTok{*} \FunctionTok{log}\NormalTok{(}\DecValTok{1} \SpecialCharTok{+}\NormalTok{ df}\SpecialCharTok{$}\NormalTok{SBUX\_simple)}
\NormalTok{df}\SpecialCharTok{$}\NormalTok{SP500\_log }\OtherTok{\textless{}{-}} \DecValTok{100} \SpecialCharTok{*} \FunctionTok{log}\NormalTok{(}\DecValTok{1} \SpecialCharTok{+}\NormalTok{ df}\SpecialCharTok{$}\NormalTok{SP500\_simple)}
\end{Highlighting}
\end{Shaded}

\subsection{🔎 ACF 与 PACF 图}\label{acf-ux4e0e-pacf-ux56fe}

\begin{Shaded}
\begin{Highlighting}[]
\NormalTok{sbux\_log }\OtherTok{\textless{}{-}} \FunctionTok{na.omit}\NormalTok{(df}\SpecialCharTok{$}\NormalTok{SBUX\_log)}

\FunctionTok{acf}\NormalTok{(sbux\_log, }\AttributeTok{lag.max =} \DecValTok{10}\NormalTok{, }\AttributeTok{main =} \StringTok{"ACF of Starbucks Log Returns"}\NormalTok{)}
\end{Highlighting}
\end{Shaded}

\pandocbounded{\includegraphics[keepaspectratio]{未命名.pdf_files/figure-latex/unnamed-chunk-3-1.pdf}}

\begin{Shaded}
\begin{Highlighting}[]
\FunctionTok{pacf}\NormalTok{(sbux\_log, }\AttributeTok{lag.max =} \DecValTok{10}\NormalTok{, }\AttributeTok{main =} \StringTok{"PACF of Starbucks Log Returns"}\NormalTok{)}
\end{Highlighting}
\end{Shaded}

\pandocbounded{\includegraphics[keepaspectratio]{未命名.pdf_files/figure-latex/unnamed-chunk-3-2.pdf}}

\subsection{🧪 Ljung-Box 检验}\label{ljung-box-ux68c0ux9a8c}

\begin{Shaded}
\begin{Highlighting}[]
\NormalTok{ljung\_result }\OtherTok{\textless{}{-}} \FunctionTok{Box.test}\NormalTok{(sbux\_log, }\AttributeTok{lag =} \DecValTok{10}\NormalTok{, }\AttributeTok{type =} \StringTok{"Ljung{-}Box"}\NormalTok{)}
\NormalTok{ljung\_result}
\end{Highlighting}
\end{Shaded}

\begin{verbatim}
## 
##  Box-Ljung test
## 
## data:  sbux_log
## X-squared = 19.823, df = 10, p-value = 0.03098
\end{verbatim}

\begin{Shaded}
\begin{Highlighting}[]
\ControlFlowTok{if}\NormalTok{ (ljung\_result}\SpecialCharTok{$}\NormalTok{p.value }\SpecialCharTok{\textless{}} \FloatTok{0.05}\NormalTok{) \{}
  \FunctionTok{cat}\NormalTok{(}\StringTok{"✅ Ljung{-}Box 检验结果:p值 ="}\NormalTok{, }\FunctionTok{round}\NormalTok{(ljung\_result}\SpecialCharTok{$}\NormalTok{p.value, }\DecValTok{4}\NormalTok{), }\StringTok{",拒绝白噪声假设,序列存在显著序列相关性。}\SpecialCharTok{\textbackslash{}n}\StringTok{"}\NormalTok{)}
\NormalTok{\} }\ControlFlowTok{else}\NormalTok{ \{}
  \FunctionTok{cat}\NormalTok{(}\StringTok{"🔍 Ljung{-}Box 检验结果:p值 ="}\NormalTok{, }\FunctionTok{round}\NormalTok{(ljung\_result}\SpecialCharTok{$}\NormalTok{p.value, }\DecValTok{4}\NormalTok{), }\StringTok{",未拒绝白噪声假设,序列看起来像白噪声。}\SpecialCharTok{\textbackslash{}n}\StringTok{"}\NormalTok{)}
\NormalTok{\}}
\end{Highlighting}
\end{Shaded}

\begin{verbatim}
## ✅ Ljung-Box 检验结果:p值 = 0.031 ,拒绝白噪声假设,序列存在显著序列相关性。
\end{verbatim}

\subsection{✅ 分析总结}\label{ux5206ux6790ux603bux7ed3}

\begin{itemize}
\tightlist
\item
  成功读取 Starbucks 股票收益率数据\\
\item
  转换为对数收益率\\
\item
  ACF 与 PACF 图揭示一阶显著自相关\\
\item
  Ljung-Box 检验支持存在序列相关性\\
\item
  ✅ 后续可进行 GARCH 建模分析
\end{itemize}

\end{document}
